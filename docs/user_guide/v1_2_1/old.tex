\subsection{Python object-orientated programming}
    The \texttt{rss\_ringoccs} package is written in Python
    in part because it is a strongly but intuitively
    object-oriented language. To that end, users who are
    not well-versed in object-oriented programming,
    particularly in a Python context, will need to
    familiarize themselves with terms associated
    with the use of objects in Python. A \textit{class}
    is a skeletal framework of functions and variables,
    the latter of which exist as either undefined or
    default properties referred to \textit{attributes}.
    When a class is called by a user (an act referred to
    as \textit{instantiation}), this creates an
    \textit{instance} of this class
    with attributes unique to this specific creation
    of the class. For more details on object-oriented
    programming, we recommend online resources such as\\
    \url{https://realpython.com/%
         python3-object-oriented-programming/}.
            \par\hfill\par
            \textbf{DiffractionCorrection Arguments}
            \begin{table}[H]
                \centering
                \begin{tabular}{lp{5in}}
                    norm\_inst:&\\
                    &An instance of the NormDiff class which
                     contains all of the necessary variables
                     to perform diffraction reconstruction.
                     This includes the geometry of the
                     occultation, the diffracted raw power,
                     and the phase.\\
                    res:&\\
                    &The user-requested resolution,
                     specified in kilometers, of the reconstructed
                     profiles. This is a positive floating point
                     number. It is bounded below by twice
                     the sample spacing that is contained in the
                     NormDiff class, in adherence to the
                     sampling theorem. Requests for smaller
                     resolutions will produce an error message.\\
                \end{tabular}
            \end{table}
            \textbf{DiffractionCorrection Keywords}
            \begin{table}[H]
                \centering
                \begin{tabular}{lp{5in}}
                    rng:&\\
                    &The requested ring plane radial range for diffraction reconstruction. This
                     can be either a list or a string, depending on
                     the planet. Preferred inputs are lists of the
                     form $\textrm{rng}=[a,b]$,
                     where $a$ is the (positive) inner limit to the radial range (in km) 
                     for reconstruction, and $b$ is the outer limit (again, in km).
                     For Saturn, several regions
                     of interest are already specified and can be
                     loaded in as strings. For example,
                     rng=``maxwell'' is a valid input and will
                     result in a reconstruction over the radial range of the Maxwell ringlet: 
                     $[87410,87610]$. Currently acceptable strings
                     are: `all', `cringripples', `encke',
                     `janusepimetheus', `maxwell', `titan', `huygens'.
                   \\
                    wtype:&\\
                    &A string specifying the requested
                     window/tapering function to be applied during the
                     reconstruction process. By default a
                     Kaiser-Bessel function with $\alpha$ parameter
                     set to 2.5 is used. The user may select any
                     of the following: `rect', `coss',
                     `kb20', `kb25', `kb35', `kbmd20', `kbmd25', which
                     stand for rectangular window,
                     squared cosine window, and various
                     Kaiser-Bessel windows. The `kbmd' windows are
                     modified versions of the Kaiser-Bessel functions.
                     The standard ones do not tend to zero at the
                     edge of the window, but rather converge to
                     $1/I_{0}(\alpha\pi)$, where $I_{0}$ is the
                     modified Bessel function.
                     For large $\alpha$ this is essentially
                     zero, but for small values this discontinuity
                     in the window can cause unwanted fringe effects.
                     The modification simply makes the function
                     zero at the edges, while still evaluating to
                     1 at the center.\\
                    fwd:&\\
                    &A Boolean for determining whether or not
                     a forward calculation will be performed at
                     the end of reconstruction. If the reconstruction
                     went well, this forward model should match the
                     data to the extent of smoothing effects caused
                     by the finite positive resolution that was
                     requested. Default is set to False.\\
    
                \end{tabular}
            \end{table}
            \newpage
            \textbf{DiffractionCorrection Keywords (Cont.)}
            \begin{table}[H]
                \centering
                \begin{tabular}{lp{5in}}
                 
                     norm:&\\
                    &A Boolean for determining whether or not to
                     normalize the reconstruction by the window width.
                     The reconstruction is defined as:
                     \begin{equation*}
                         T(\rho)
                         =\frac{1-i}{2F}
                          \int_{-W/2}^{W/2}\hat{T}(\rho_{0})
                          w(\rho-\rho_{0})e^{-i\psi(\rho,\rho_{0})}
                          d\rho_{0},
                     \end{equation*}
                     where $\hat{T}$ is the diffracted data,
                     $\exp(-i\psi)$ is the Fresnel Kernel,
                     $w$ is the window function, and $W$ is the
                     window width. A quick look at
                     this equation shows that as the window width
                     goes to zero, so does the reconstruction. That
                     means for large resolutions (10-km, or so),
                     the reconstruction will be a small fraction
                     of the input data. To correct this
                     we offer the ability to normalize the
                     reconstruction by the following factor:
                     \begin{equation*}
                         N
                         =\frac{\int_{-\infty}^{\infty}e^{i\psi}d\rho}
                          {\int_{-W/2}^{W/2}we^{i\psi}d\rho}
                     \end{equation*}
                     Default is set to True.\\
                    bfac:&\\
                    &A Boolean for determining how to compute the
                     window width about a particular point. The
                     standard definition of the window width is
                     $W=2F^{2}/\Delta{R}$,
                     where $F$ is the Fresnel Scale and $\Delta{R}$ is
                     the requested resolution. However, when the
                     Allen deviation is large, or when the velocity
                     of ring intercept point ($d\rho/dt$) is small,
                     the resolution is defined as:
                     \begin{equation*}
                         \Delta{R}
                         =\frac{2F^{2}}{W}\frac{b^2/2}{\exp(-b)+b-1},
                     \end{equation*}
                    where $b=\omega^{2}\sigma^{2}W/2\dot{\rho}$.
                     Here, $\omega$ is the angular frequency,
                     $\sigma$ is the Allen deviation, and $\dot{\rho}$
                     is the ring intercept velocity. Solving
                     for $W$ then becomes a process of inverting this
                     equation, which we perform using LambertW functions.
                     Setting bfac=True will compute the window width
                     taking into account this ``b'' factor, setting
                     bfac=False uses the standard $W=2F^{2}/\Delta{R}$.
                     Default is True. For all Cassini RSS observations that used the USO, the Allen deviation is sufficiently small that $b<<1$.\\
                    sigma:&\\
                    &The Allen deviation. This is a positive floating
                     point number. If bfac=False is set, this parameter
                     is ignored. See bfac for more documentation.\\
                    fft:&\\
                     &A Boolean for determining whether or not to use
                      FFT's for the computation. When the Fresnel kernel
                      (see the equation listed under the ``norm''
                      keyword) is of the form
                      $\exp(i\psi(\rho-\rho_{0}))$, the convolution
                      theorem may be applied to compute the
                      reconstruction. This can greatly reduce
                      computation time. Default is False.\\
                    psitype:&\\
                    &A string that will determine how the Fresnel kernel
                     will be approximated (see the equation listed under
                     then ``norm'' keyword). Currently there are several
                     options: `full', `mtr2', `mtr3', `mtr4', and
                     'fresnel', none of which are case-sensitive.
                     The MTR options compute the polynomial
                     approximations found in MTR86.
                     ``Full'' uses no approximation and $\psi$ will be
                     computed outright. ``Fresnel'' uses the
                     classic quadratic Fresnel approximation,
                     which sets
                     $\psi=\frac{\pi}{2}(\frac{\rho-\rho_{0}}{F})^{2}$.
                     Because of the simplicity of the computation,
                     `fresnel' is by far the fastest in
                     reconstruction time. For many
                     occultations, such as Cassini's rev007,
                     the difference between using `fresnel' and `full'
                     is extremely small.
                     For more pathological
                     data sets, such as rev133 (with a very small ring opening angle), there are significant
                     differences. The user must be alert to note when
                     various approximations are valid.
                     Default is psitype=`full'.\\
                    write\_file:&\\
                    &A Boolean for determining if the *.TAU file will
                     be written.\\
                    verbose:&\\
                    &Boolean for printing out status reports on
                     the reconstruction. Default is False.
                \end{tabular}
            \end{table}